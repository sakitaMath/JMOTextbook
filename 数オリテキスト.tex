\documentclass[uplatex,fleqn]{jsbook}
\usepackage{amsmath,amssymb,amsthm}
\begin{document}

\title{数オリテキスト(仮)}
\author{佐世保北高校数学オリンピック勉強会}
\date{令和2年度}
\maketitle

\chapter{はじめに}
\section{記号についての説明}
\subsection{集合の記号}


\chapter{代数}
\section{方程式}
\subsection{同じ部分をまとめる}
同じ部分はいったんまとめる事で、式がすっきりして解きやすくなることがあります。
同じ部分を見つけたら、文字で置くなどしてまとめましょう。
また、文字で置いた場合は値の範囲も確認して、ありえない値を書かないようにしょう。

\vspace{15pt} {\large \textbf{例題}}
\begin{quote}
    $(x^2+2x)^2+3(x^2+2x)+2=0$

    $X=x^2+2x$とすると($X\geqq-1$ --- ①)

    $(X+1)(X+2)=0$

    ①より

    $X=-1$

    $x^2+2x=-1$

    $x^2+2x+1=0$

    $(x+1)^2=0$

    $x=-1$
\end{quote}

\vspace{15pt}{\large \textbf{練習問題}}
\begin{quote}
    次の方程式を解け。
    $\displaystyle \frac{1}{x^2-10x-29}+\frac{1}{x^2-10x-45}-\frac{2}{x^2-10x-69}=0$
\end{quote}

\vspace{15pt}\fbox{\textbf{ポイント}}

$x^2-10x$に着目し、まとめる。

\vspace{15pt} {\large \textbf{解答}}
\begin{quote}
    $X=x^2-10x-49$とおくと($X\geqq-74$ --- ①)

    $\displaystyle \frac{1}{X+20}+\frac{1}{X+4}-\frac{2}{X-20}=0$

    $(X+4)(X-20)+(X+20)(X-20)-2(X+20)(X+4)=0$

    $X-64X-640=0$

    $X=-10$

    ①より適

    $x^2-10x-49=-10$

    $x^2-10x-39=0$

    $(x-13)(x+3)=0$

    $x=-3,13$
\end{quote}

\subsection{解と係数の関係}

$ax^2+bx+c=0$
の解を
$x=\alpha,\beta$
とすると
$(a\neq0)$
\begin{quote}
    $\displaystyle \alpha+\beta=-\frac{b}{a}$

    $\displaystyle \alpha\beta=\frac{c}{a}$
\end{quote}
因数分解をしたときに、
\begin{quote}
    $ax^2+bx+c=a(x-\alpha)(x-\beta)$
\end{quote}
となリます。右辺を展開し、係数を比較することで導くことができます。

2次方程式だけでなく、3次以上の場合でも、上と同じように考えることができます。

\vspace{15pt}{\large \textbf{練習問題1}}
\begin{quote}
    $x^{1995}-x+5=0$

    の全ての解の1995乗の和を求めよ。
\end{quote}

\vspace{15pt} {\large \textbf{解答}}
\begin{quote}
    解を$x_1,x_2,x_3,\dots,x_{1995}$とすると、

    $x^{1995}-x+5=(x-x_1)(x-x_2)(x-x_3)\dots (x-x_{1995})$

    $x^{1994}$の係数を比較すると、

    左辺では$0$

    右辺では$-(x_1+x_2+x_3+\dots +x_{1995})$であるから、

    $x_1+x_2+x_3+\dots +x_{1995}=0$

    $x^{1995}-x+5=0$

    $x^{1995}=x-5$

    よって全ての解の1995乗の和は、全ての解からそれぞれ5を引いたものの和に等しいから、

    $(x_1-5)+(x_2-5)+(x_3-5)+\dots +(x_{1995}-5)$

    $=x_1+x_2+x_3+\dots +x_{1995}-5\times 1995$

    $=x_1+x_2+x_3+\dots +x_{1995}-5\times 1995$

    $=-5\times 1995$

    $=-9975$
\end{quote}
$n$次方程式の$x^{n-1}$の係数が、解の総和に$-1$をかけたものに等しいことは、覚えておいても良いでしょう。

\vspace{15pt}{\large \textbf{練習問題2}}
\begin{quote}
    次の方程式を解け。

    \begin{math}
        \begin{cases}
            xy+x+y=71\\
            x^2y+xy^2=88
        \end{cases}
        (x,y\in\mathbb{N})
    \end{math}
\end{quote}

\vspace{15pt}{\large \textbf{解答}}
\begin{quote}
    x,yは整数なので因数分解をすることで解くこともできるが、解と係数の関係を使うと、約数を全通り試す必要なく解くことができる。

    \begin{math}
        \begin{cases}
            xy+x+y=71\\
            xy(x+y)=880
        \end{cases}
    \end{math}

    $a$に関する2次関数$a^2-71a+88=0$は、$xy$と$x+y$を解にもつ

    $a^2-71a+88=0$

    $(a-16)(a-55)=0$

    $a=16,55$

    $\rm(\hspace{.18em}i\hspace{.18em})$$x+y=16$, $xy=55$のとき

    \begin{quote}
        $b$に関する2次関数$b^2-16b+55=0$は、$x$と$y$を解にもつ

        $b^2-16b+55=0$

        $(b-5)(b-11)=0$

        $b=5,11$

        よって$(x,y)=(5,11),(11,5)$
    \end{quote}

    $\rm(\hspace{.08em}ii\hspace{.08em})$$x+y=55$, $xy=16$のとき
    \begin{quote}
        $b$に関する2次関数$b^2-55b+16=0$は、$x$と$y$を解にもつ

        $b^2-55b+16=0$

        $\displaystyle b=\frac{-3 \; \sqrt{329} + 55}{2}, b=\frac{3 \; \sqrt{329} + 55}{2}$

        $x,y\in\mathbb{N}$より不適
    \end{quote}

    $\therefore (x,y)=(5,11),(11,5)$
\end{quote}

\section{不等式}
\subsection{2乗をつくる}
実数は2乗をすると0以上になる
不等式の証明をするときは、方針として、以下のような形を作る

$A^2\geqq 0$

$B^2\geqq 0$

$A^2+B^2\geqq 0$

\vspace{15pt}{\large \textbf{練習問題}}
\begin{quote}
    $a^2+b^2+c^2\geqq ab+bc+ca$を示せ。
\end{quote}

\vspace{15pt}{\large \textbf{解答}}
\begin{quote}
    方針として、$(a- b)^2=a^2-2ab+b^2$を使うために、$2ab$をつくる。

    $(\text{左辺})-(\text{右辺})$

    $=a^2+b^2+c^2-ab-bc-ca$

    $\displaystyle =\frac{1}{2}(2a^2+2b^2+2c^2-2ab-2bc-2ca)$

    $\displaystyle =\frac{1}{2}(a^2-2ab+b^2+b^2-2bc+c^2+c^2-2ca+a^2)$

    $\displaystyle =\frac{1}{2}\{(a-b)^2+(b-c)^2+(c-a)^2\}\geqq 0$

    \qed
\end{quote}

\subsection{相加相乗平均}
$\displaystyle \frac{a+b}{2}$を相加平均といい、$\sqrt{ab}$を相乗平均という。

$a,b>0$のとき、$a+b\geqq2\sqrt{ab}$が成り立つ。

また3変数以上にも拡張することができる。

3変数の場合$a+b+c\geqq 3\sqrt[3]{abc}\ (a,b,c>0)$

n変数の場合$\displaystyle \sum_{i=1}^n a_i\geqq n\sqrt[n]{\prod_{i=1}^n a_i}\ (a_1,a_2,a_3,\dots,a_n)$

また、$\displaystyle \frac{2}{\frac{1}{a}+\frac{1}{b}}$を調和平均という。逆数の相加平均の逆数である。

相加平均$\geqq$相乗平均$\geqq$調和平均が成り立つ。このことは相加相乗平均の不等式から導ける。
\begin{quote}
    \begin{proof}
        相加平均と相乗平均の逆数を取ると、

        $\displaystyle \frac{1}{\sqrt{ab}}\geqq\frac{2}{a+b}$

        $\displaystyle \sqrt{ab}\geqq\frac{2ab}{a+b}$

        $\displaystyle \sqrt{ab}\geqq\frac{2}{\frac{1}{a}+\frac{1}{b}}$
    \end{proof}
\end{quote}

等号成立は、$a=b$のときである。

\vspace{15pt}{\large \textbf{練習問題}}
\begin{quote}
    $x,y,z>0$とする

    $\displaystyle \frac{x^3y^2z}{x^6+y^6+z^6}$の最大値を求めよ。
\end{quote}

\vspace{15pt}{\large \textbf{解答}}
\begin{align*}
    &\text{分母が最小になるときに最大になる}\\
    &x^6+y^6+z^6\\
    &=\frac{1}{3}x^6+\frac{1}{3}x^6+\frac{1}{3}x^6+\frac{1}{2}y^6+\frac{1}{2}y^6+z^6\text{ ---①}\\
    \text{①}&\geqq6\sqrt[6]{\frac{1}{3}x^6\times\frac{1}{3}x^6\times\frac{1}{3}x^6\times\frac{1}{2}y^6\times\frac{1}{2}y^6\times z^6}\\
    &=6\sqrt[6]{\frac{1}{108}}\ x^3y^2z\\
    &=\sqrt[3]{4}\times\sqrt[2]{3}x^3y^2z\\
    &\therefore \frac{1}{\sqrt[3]{4}\times\sqrt[2]{3}}\\
    &\text{①のような変形をすることで無理矢理}x^3y^2z\times{をつくって消すことができた。}
\end{align*}

\subsection{コーシーシュワルツ不等式}
$\displaystyle (\sum_{i=1}^n a^2_i)(\sum_{i=1}^n b^2_i)\geqq(\sum_{i=1}^n a_ib_i)^2$つまり

$(a^2_1,a^2_2,a^2_3\dots,a^2_n)(b^2_1,b^2_2,b^2_3\dots,b^2_n)\geqq(a_1b_1,a_2b_2,a_3b_3,\dots,a_nb_n)^2$

が成り立つ。
\begin{quote}
    \begin{proof}
        $n$次元のベクトルの内積について考える

        $\vec{a}=(a_1,a_2,a_3,\dots,a_n)$

        $\vec{b}=(b_1,b_2,b_3,\dots,b_n)$

        とする

        $\vec{a}\cdot\vec{b}=|\vec{a}||\vec{b}|\cos\theta$

        $(\vec{a}\cdot\vec{b})^2=|\vec{a}|^2|\vec{b}|^2\cos^2\theta$

        $\cos^2\theta\leqq1$より、

        $|\vec{a}|^2|\vec{b}|^2\geqq(\vec{a}\cdot\vec{b})^2$

        $(a^2_1,a^2_2,a^2_3\dots,a^2_n)(b^2_1,b^2_2,b^2_3\dots,b^2_n)\geqq(a_1b_1,a_2b_2,a_3b_3,\dots,a_nb_n)^2$

        等号成立条件は、$\cos^2\theta=1$すなわち2つのベクトルが並行なときであり、

        $a_1:a_2:a_3:\dots:a_n=b_1:b_2:b_3:\dots:b_n$と同値である。
    \end{proof}
\end{quote}

\vspace{15pt}{\large \textbf{例題}}
\begin{quote}
    $4(w^2+x^2+y^2+z^2)\geqq (w+x+y+z)^2$を示す。

    $(\text{左辺})=(1+1+1+1)(w^2+x^2+y^2+z^2)$

    コーシーシュワルツ不等式より、

    $(1+1+1+1)(w^2+x^2+y^2+z^2)\geqq (w+x+y+z)^2$

    よって$(\text{左辺})\geqq (右辺)$となり、成り立つ。

    等号成立は$w:x:y:z=1:1:1:1$

    つまり$w=x=y=z$のとき。
\end{quote}

\paragraph{注}有名不等式を使うときは、名前を書くこと。

\vspace{15pt}{\large \textbf{練習問題}}
\begin{quote}
    $x+y+z=1$、$x,y,z>0$のとき

    $\displaystyle \frac{1}{x}+\frac{4}{y}+\frac{9}{z}$の最小値を求めよ。
\end{quote}

\vspace{15pt}{\large \textbf{解答}}
\begin{quote}
    コーシーシュワルツ不等式より、

    $\displaystyle (x+y+z)(\frac{1}{x}+\frac{4}{y}+\frac{9}{z})\geqq (1+2+3)^2$

    等号成立条件は、

    $\displaystyle x:y:z=\frac{1}{x}:\frac{4}{y}:\frac{9}{z}$

    $x^2=\frac{y^2}{4}=\frac{z^2}{9}$

    $x,y,z>0$より、

    $x=\frac{y}{2}=\frac{z}{3}$

    $\therefore 36$
\end{quote}
\end{document}