\documentclass[uplatex,fleqn]{jsbook}
\usepackage{amsmath,amssymb,amsthm}
\begin{document}

\title{数オリテキスト(仮)}
\author{佐世保北高校数学オリンピック勉強会}
\date{令和2年度}
\maketitle

\chapter{はじめに}
\section{必要な知識}
\section{記号・用語についての説明}
\subsection{集合の記号}
\begin{table}[h]
    \begin{tabular}{ll}
        記号 & 意味\\\hline\hline
        $\mathbb{N}$ & 自然数\\
        $\mathbb{Z}$ & 整数\\
        $\mathbb{Q}$ & 有理数\\
        $\mathbb{R}$ & 実数\\
        $\mathbb{C}$ & 複素数\\\hline
    \end{tabular}
\end{table}
\subsection{}

\chapter{代数}
\section{方程式}
\subsection{同じ部分をまとめる}
同じ部分はいったんまとめる事で、式がすっきりして解きやすくなることがあるので、
同じ部分を見つけたら、文字で置くなどしてまとめる。
また、文字で置いた場合は値の範囲も確認して、ありえない値を書かないようにすること。

\vspace{15pt} {\large \textbf{例題}}
\begin{quote}
    $\left(x^2+2x\right)^2+3\left(x^2+2x\right)+2=0$

    $X=x^2+2x$とすると($X\geqq-1$ --- ①)

    $\left(X+1\right)\left(X+2\right)=0$

    ①より

    $X=-1$

    $x^2+2x=-1$

    $x^2+2x+1=0$

    $\left(x+1\right)^2=0$

    $x=-1$
\end{quote}

\vspace{15pt}{\large \textbf{練習問題}}
\begin{quote}
    次の方程式を解け。
    $\displaystyle \frac{1}{x^2-10x-29}+\frac{1}{x^2-10x-45}-\frac{2}{x^2-10x-69}=0$
\end{quote}

\paragraph{ポイント}$x^2-10x$に着目し、まとめる。

\vspace{15pt} {\large \textbf{解答}}
\begin{quote}
    $X=x^2-10x-49$とおくと($X\geqq-74$ --- ①)

    $\displaystyle \frac{1}{X+20}+\frac{1}{X+4}-\frac{2}{X-20}=0$

    $\left(X+4\right)\left(X-20\right)+\left(X+20\right)\left(X-20\right)-2\left(X+20\right)\left(X+4\right)=0$

    $X-64X-640=0$

    $X=-10$

    ①より適

    $x^2-10x-49=-10$

    $x^2-10x-39=0$

    $\left(x-13\right)\left(x+3\right)=0$

    $x=-3,13$
\end{quote}

\subsection{解と係数の関係}

$ax^2+bx+c=0$
の解を
$x=\alpha,\beta$
とすると
$\left(a\neq0\right)$
\begin{quote}
    $\displaystyle \alpha+\beta=-\frac{b}{a}$

    $\displaystyle \alpha\beta=\frac{c}{a}$
\end{quote}
因数分解をしたときに、
\begin{quote}
    $ax^2+bx+c=a\left(x-\alpha\right)\left(x-\beta\right)$
\end{quote}
となリます。右辺を展開し、係数を比較することで導くことができる。

2次方程式だけでなく、3次以上の場合でも、上と同じように考えることができる。

\vspace{15pt}{\large \textbf{練習問題1}}
\begin{quote}
    $x^{1995}-x+5=0$

    の全ての解の1995乗の和を求めよ。
\end{quote}

\vspace{15pt} {\large \textbf{解答}}
\begin{quote}
    解を$x_1,x_2,x_3,\dots,x_{1995}$とすると、

    $x^{1995}-x+5=\left(x-x_1\right)\left(x-x_2\right)\left(x-x_3\right)\dots \left(x-x_{1995}\right)$

    $x^{1994}$の係数を比較すると、

    左辺では$0$

    右辺では$-\left(x_1+x_2+x_3+\dots +x_{1995}\right)$であるから、

    $x_1+x_2+x_3+\dots +x_{1995}=0$

    $x^{1995}-x+5=0$

    $x^{1995}=x-5$

    よって全ての解の1995乗の和は、全ての解からそれぞれ5を引いたものの和に等しいから、

    $\left(x_1-5\right)+\left(x_2-5\right)+\left(x_3-5\right)+\dots +\left(x_{1995}-5\right)$

    $=x_1+x_2+x_3+\dots +x_{1995}-5\times 1995$

    $=x_1+x_2+x_3+\dots +x_{1995}-5\times 1995$

    $=-5\times 1995$

    $=-9975$
\end{quote}
$n$次方程式の$x^{n-1}$の係数が、解の総和に$-1$をかけたものに等しいことは、覚えておいても良い。

\vspace{15pt}{\large \textbf{練習問題2}}
\begin{quote}
    次の方程式を解け。

    \begin{math}
        \begin{cases}
            xy+x+y=71\\
            x^2y+xy^2=88
        \end{cases}
        \left(x,y\in\mathbb{N}\right)
    \end{math}
\end{quote}

\vspace{15pt}{\large \textbf{解答}}
\begin{quote}
    x,yは整数なので因数分解をすることで解くこともできるが、解と係数の関係を使うと、約数を全通り試す必要なく解くことができる。

    \begin{math}
        \begin{cases}
            xy+x+y=71\\
            xy\left(x+y\right)=880
        \end{cases}
    \end{math}

    $a$に関する2次関数$a^2-71a+88=0$は、$xy$と$x+y$を解にもつ

    $a^2-71a+88=0$

    $\left(a-16\right)\left(a-55\right)=0$

    $a=16,55$

    $\rm\left(\hspace{.18em}i\hspace{.18em}\right)$$x+y=16$, $xy=55$のとき

    \begin{quote}
        $b$に関する2次関数$b^2-16b+55=0$は、$x$と$y$を解にもつ

        $b^2-16b+55=0$

        $\left(b-5\right)\left(b-11\right)=0$

        $b=5,11$

        よって$\left(x,y\right)=\left(5,11\right),\left(11,5\right)$
    \end{quote}

    $\rm\left(\hspace{.08em}ii\hspace{.08em}\right)$$x+y=55$, $xy=16$のとき
    \begin{quote}
        $b$に関する2次関数$b^2-55b+16=0$は、$x$と$y$を解にもつ

        $b^2-55b+16=0$

        $\displaystyle b=\frac{-3 \; \sqrt{329} + 55}{2}, b=\frac{3 \; \sqrt{329} + 55}{2}$

        $x,y\in\mathbb{N}$より不適
    \end{quote}

    $\therefore \left(x,y\right)=\left(5,11\right),\left(11,5\right)$
\end{quote}

\section{不等式}
\subsection{2乗をつくる}
実数は2乗をすると0以上になる
不等式の証明をするときは、方針として、以下のような形を作る

$A^2\geqq 0$

$B^2\geqq 0$

$A^2+B^2\geqq 0$

\vspace{15pt}{\large \textbf{練習問題}}
\begin{quote}
    $a^2+b^2+c^2\geqq ab+bc+ca$を示せ。
\end{quote}

\vspace{15pt}{\large \textbf{解答}}
\begin{quote}
    方針として、$\left(a- b\right)^2=a^2-2ab+b^2$を使うために、$2ab$をつくる。

    $\left(\text{左辺}\right)-\left(\text{右辺}\right)$

    $=a^2+b^2+c^2-ab-bc-ca$

    $\displaystyle =\frac{1}{2}\left(2a^2+2b^2+2c^2-2ab-2bc-2ca\right)$

    $\displaystyle =\frac{1}{2}\left(a^2-2ab+b^2+b^2-2bc+c^2+c^2-2ca+a^2\right)$

    $\displaystyle =\frac{1}{2}\{\left(a-b\right)^2+\left(b-c\right)^2+\left(c-a\right)^2\}\geqq 0$

    \qed
\end{quote}

\subsection{相加相乗平均}
$\displaystyle \frac{a+b}{2}$を相加平均といい、$\sqrt{ab}$を相乗平均という。

$a,b>0$のとき、$a+b\geqq2\sqrt{ab}$が成り立つ。

また3変数以上にも拡張することができる。

3変数の場合$a+b+c\geqq 3\sqrt[3]{abc}\ \left(a,b,c>0\right)$

n変数の場合$\displaystyle \sum_{i=1}^n a_i\geqq n\sqrt[n]{\prod_{i=1}^n a_i}\ \left(a_1,a_2,a_3,\dots,a_n\right)$

また、$\displaystyle \frac{2}{\frac{1}{a}+\frac{1}{b}}$を調和平均という。逆数の相加平均の逆数である。

相加平均$\geqq$相乗平均$\geqq$調和平均が成り立つ。このことは相加相乗平均の不等式から導ける。
\begin{quote}
    \begin{proof}
        相加平均と相乗平均の逆数を取ると、

        $\displaystyle \frac{1}{\sqrt{ab}}\geqq\frac{2}{a+b}$

        $\displaystyle \sqrt{ab}\geqq\frac{2ab}{a+b}$

        $\displaystyle \sqrt{ab}\geqq\frac{2}{\frac{1}{a}+\frac{1}{b}}$
    \end{proof}
\end{quote}

等号成立は、$a=b$のときである。

\vspace{15pt}{\large \textbf{練習問題}}
\begin{quote}
    $x,y,z>0$とする

    $\displaystyle \frac{x^3y^2z}{x^6+y^6+z^6}$の最大値を求めよ。
\end{quote}

\vspace{15pt}{\large \textbf{解答}}
\begin{align*}
    &\text{分母が最小になるときに最大になる}\\
    &x^6+y^6+z^6\\
    &=\frac{1}{3}x^6+\frac{1}{3}x^6+\frac{1}{3}x^6+\frac{1}{2}y^6+\frac{1}{2}y^6+z^6\text{ ---①}\\
    \text{①}&\geqq6\sqrt[6]{\frac{1}{3}x^6\times\frac{1}{3}x^6\times\frac{1}{3}x^6\times\frac{1}{2}y^6\times\frac{1}{2}y^6\times z^6}\\
    &=6\sqrt[6]{\frac{1}{108}}\ x^3y^2z\\
    &=\sqrt[3]{4}\times\sqrt[2]{3}x^3y^2z\\
    &\therefore \frac{1}{\sqrt[3]{4}\times\sqrt[2]{3}}\\
    &\text{①のような変形をすることで無理矢理}x^3y^2z\times{をつくって消すことができた。}
\end{align*}

\subsection{コーシーシュワルツ不等式}
$\displaystyle \left(\sum_{i=1}^n a^2_i\right)\left(\sum_{i=1}^n b^2_i\right)\geqq\left(\sum_{i=1}^n a_ib_i\right)^2$つまり

$\left(a^2_1,a^2_2,a^2_3\dots,a^2_n\right)\left(b^2_1,b^2_2,b^2_3\dots,b^2_n\right)\geqq\left(a_1b_1,a_2b_2,a_3b_3,\dots,a_nb_n\right)^2$

が成り立つ。
\begin{quote}
    \begin{proof}
        $n$次元のベクトルの内積について考える

        $\vec{a}=\left(a_1,a_2,a_3,\dots,a_n\right)$

        $\vec{b}=\left(b_1,b_2,b_3,\dots,b_n\right)$

        とする

        $\vec{a}\cdot\vec{b}=|\vec{a}||\vec{b}|\cos\theta$

        $\left(\vec{a}\cdot\vec{b}\right)^2=|\vec{a}|^2|\vec{b}|^2\cos^2\theta$

        $\cos^2\theta\leqq1$より、

        $|\vec{a}|^2|\vec{b}|^2\geqq\left(\vec{a}\cdot\vec{b}\right)^2$

        $\left(a^2_1,a^2_2,a^2_3\dots,a^2_n\right)\left(b^2_1,b^2_2,b^2_3\dots,b^2_n\right)\geqq\left(a_1b_1,a_2b_2,a_3b_3,\dots,a_nb_n\right)^2$

        等号成立条件は、$\cos^2\theta=1$すなわち2つのベクトルが並行なときであり、

        $a_1:a_2:a_3:\dots:a_n=b_1:b_2:b_3:\dots:b_n$と同値である。
    \end{proof}
\end{quote}

\vspace{15pt}{\large \textbf{例題}}
\begin{quote}
    $4\left(w^2+x^2+y^2+z^2\right)\geqq \left(w+x+y+z\right)^2$を示す。

    $\left(\text{左辺}\right)=\left(1+1+1+1\right)\left(w^2+x^2+y^2+z^2\right)$

    コーシーシュワルツ不等式より、

    $\left(1+1+1+1\right)\left(w^2+x^2+y^2+z^2\right)\geqq \left(w+x+y+z\right)^2$

    よって$\left(\text{左辺}\right)\geqq \left(右辺\right)$となり、成り立つ。

    等号成立は$w:x:y:z=1:1:1:1$

    つまり$w=x=y=z$のとき。
\end{quote}

\paragraph{注}有名不等式を使うときは、名前を書くこと。

\vspace{15pt}{\large \textbf{練習問題}}
\begin{quote}
    $x+y+z=1$、$x,y,z>0$のとき

    $\displaystyle \frac{1}{x}+\frac{4}{y}+\frac{9}{z}$の最小値を求めよ。
\end{quote}

\vspace{15pt}{\large \textbf{解答}}
\begin{quote}
    コーシーシュワルツ不等式より、

    $\displaystyle \left(x+y+z\right)\left(\frac{1}{x}+\frac{4}{y}+\frac{9}{z}\right)\geqq \left(1+2+3\right)^2$

    等号成立条件は、

    $\displaystyle x:y:z=\frac{1}{x}:\frac{4}{y}:\frac{9}{z}$

    $x^2=\frac{y^2}{4}=\frac{z^2}{9}$

    $x,y,z>0$より、

    $x=\frac{y}{2}=\frac{z}{3}$

    $\therefore 36$
\end{quote}

\section{多項式}
割り算頑張る。

\section{関数方程式}
$f\left(x+1\right)=f\left(x\right)+1$的なやつ。

\section{数列}
一般項を求める問題などが出題される。
実験$\rightarrow$予想$\rightarrow$の順で解いていく。手を動かして頑張ろう。

\vspace{15pt}{\large \textbf{練習問題1}}
\begin{quote}
    $n\in \mathbb{N}$

    $a_n:\sqrt{n}$に最も近い自然数

    $b_n=a_n+n$

    $b_n$の規則性を予想せよ。
\end{quote}

\vspace{15pt}{\large \textbf{解答}}
\begin{quote}
    とりあえず実験をしてみる。$n\leqq 20$で試してみると、以下の表のようになる。
    % \begin{table}[h]
    %     \begin{tabular}{rrr}
    %         \hline
    %         $n$ & $a_n$ & $b_n$ \\ \hline \hline
    %         $1$ & $1$ & $2$ \\
    %         $2$ & $1$ & $3$ \\
    %         $3$ & $2$ & $5$ \\
    %         $4$ & $2$ & $6$ \\
    %         $5$ & $2$ & $7$ \\
    %         $6$ & $2$ & $8$ \\
    %         $7$ & $3$ & $10$ \\
    %         $8$ & $3$ & $11$ \\
    %         $9$ & $3$ & $12$ \\
    %         $10$ & $3$ & $13$ \\
    %         $11$ & $3$ & $14$ \\
    %         $12$ & $3$ & $15$ \\
    %         $13$ & $4$ & $17$ \\
    %         $14$ & $4$ & $18$ \\
    %         $15$ & $4$ & $19$ \\
    %         $16$ & $4$ & $20$ \\
    %         $17$ & $4$ & $21$ \\
    %         $18$ & $4$ & $22$ \\
    %         $19$ & $4$ & $23$ \\
    %         $20$ & $4$ & $24$ \\\hline
    %     \end{tabular}
    % \end{table}

    \begin{table}[h]
        \begin{tabular}{|c||r|r|r|r|r|r|r|r|r|r|r|r|r|r|r|r|r|r|r|r|}
            \hline
            $n$ & $1$ & $2$ & $3$ & $4$ & $5$ & $6$ & $7$ & $8$ & $9$ & $10$ & $11$ & $12$ & $13$ & $14$ & $15$ & $16$ & $17$ & $18$ & $19$ & $20$\\\hline
            $a_n$ & $1$ & $1$ & $2$ & $2$ & $2$ & $2$ & $3$ & $3$ & $3$ & $3$ & $3$ & $3$ & $4$ & $4$ & $4$ & $4$ & $4$ & $4$ & $4$ & $4$\\\hline
            $b_n$ & $2$ & $3$ & $5$ & $6$ & $7$ & $8$ & $10$ & $11$ & $12$ & $13$ & $14$ & $15$ & $17$ & $18$ & $19$ & $20$ & $21$ & $22$ & $23$ & $24$\\\hline
        \end{tabular}
    \end{table}

    $b_n$に存在しない自然数を並べてみると、
    $1,4,9,16\dots$となっており、
    $b_n$には平方数が存在しないようだということがわかる。

    $b_n$には平方数が含まれないことを証明する。

    \begin{proof}
        $a_n=m$とすると、

        $\displaystyle \left(m-\frac{1}{2}\right) ^2\leqq n\leqq \left(m+\frac{1}{2}\right) ^2$

        $\displaystyle m^2-m+\frac{1}{4}\leqq n \leqq m^2+m+\frac{1}{4}$

        $\displaystyle m^2-m+1\leqq n \leqq m^2+m \left(\because n,m \in \mathbb{N}\right)$

        $m^2+1\leqq m+m \leqq m^2+2m$

        $m^2+1\leqq b_n \leqq \left(m+1\right)^2-1$

    \end{proof}

\end{quote}

\vspace{15pt}{\large \textbf{練習問題2}}
\begin{quote}
    \begin{math}
        \begin{cases}
            f\left(0\right)=0\\
            f\left(1\right)=1\\
            \displaystyle f\left(n\right)=f\left(\left[\frac{n}{2}\right]\right)+n-2\left[\frac{n}{2}\right]
        \end{cases}
    \end{math}

    $f\left(n\right)$の規則性を予想せよ。
\end{quote}

\vspace{15pt}{\large \textbf{解答}}
\begin{quote}
    $n\leqq 20$で試してみると、以下の表のようになる。

    \begin{table}[h]
        \begin{tabular}{|c||r|r|r|r|r|r|r|r|r|r|r|r|r|r|r|r|r|r|r|r|r|}
            \hline
            $n$ & $0$ & $1$ & $2$ & $3$ & $4$ & $5$ & $6$ & $7$ & $8$ & $9$ & $10$ & $11$ & $12$ & $13$ & $14$ & $15$ & $16$ & $17$ & $18$ & $19$ & $20$\\\hline
            $f\left(n\right)$ & $0$ & $1$ & $1$ & $2$ & $1$ & $2$ & $2$ & $3$ & $1$ & $2$ & $2$ & $3$ & $2$ & $3$ & $3$ & $4$ & $1$ & $2$ & $2$ & $3$ & $2$\\\hline
        \end{tabular}
    \end{table}

    $f\left(n\right)$は$n$を2進数で表したときの1の数である。

    $\displaystyle f\left(\left[\frac{n}{2}\right]\right)$は1の位以外の部分での1の数、$\displaystyle n-2\left[\frac{n}{2}\right]$は1の位の数を表している。
\end{quote}
\paragraph{豆知識}$\displaystyle \left[\frac{n}{a^k}\right]$は、$n$を$a$進数表記し、下$k$桁を切り落としたものになる。
$\displaystyle n-\left[\frac{n}{a^k}\right]$は、$n$を$a$進数表記したときの下$k$桁を表す。
\end{document}