\documentclass[dvipdfmx]{jsbook}

\begin{document}

\title{数オリテキスト(仮)}
\author{佐世保北高校数学オリンピック勉強会}
\date{令和2年度}
\maketitle

\part{はじめに}

\part{代数}
\chapter{方程式}
\section{同じ部分をまとめる}
同じ部分はいったんまとめる事で、式がすっきりして解きやすくなることがあります。
同じ部分を見つけたら、文字で置くなどしてまとめましょう。
また、文字で置いた場合は値の範囲も確認して、ありえない値を書かないようにしょう。

\vspace{15pt} {\large \textbf{例}}

\begin{quote}
    $(x^2+2x)^2+3(x^2+2x)+2=0$

    $X=x^2+2x$
    とすると
    $(X\geq-1$ –– $ \textcircled{\scriptsize1})$
    
    $(X+1)(X+2)=0$
    
    $ \textcircled{\scriptsize1} $より
    
    $X=-1$
    
    $x^2+2x=-1$
    
    $x^2+2x+1=0$
    
    $(x+1)^2=0$
    
    $x=-1$
\end{quote}

練習問題

$\frac{1}{x^2+-10x-29}+\frac{1}{x^2+-10x-45}-\frac{1}{x^2+-10x-69}=0$

\fbox{ポイント}

$x^2-10x$に着目し、まとめる

\paragraph{}解答

$X=x^2-10x-45$とおくと($X\geq-70$ー $\textcircled{\scriptsize1}$)

$\frac{1}{X+16}+\frac{1}{X}-\frac{1}{X-24}=0$

$X(X-24)+(X+16)(X-24)-(X+16)X=0$

$X^2-48X-384$

\section{解と係数の関係}

$ax^2+bx+c=0$
の解を
$x=\alpha,\beta$
とすると
$(a\neq0)$

$\alpha+\beta=-\frac{b}{a}$

$\alpha\beta=\frac{c}{a}$
\end{document}